\documentclass[twocolumn]{article}

\usepackage[utf8]{inputenc}
\usepackage[backend=bibtex]{biblatex} % References
\usepackage[hidelinks]{hyperref} % Clickable links
\usepackage{url} % Wrap urls
\usepackage{siunitx} % Typeset units

\sisetup{per-mode=symbol} % (siunitx) Use symbols for 'per' values e.g. m/s

\newcommand{\hPa}{\hecto\pascal} % Shorter command for hPa

\bibliography{refs.bib}

\title{Using Smart Devices as Tools in Safety Critical Skydiving Applications}
\author{Christopher Lane}

\begin{document}
\maketitle

\begin{abstract} % "Helps decide if someone should read the paper"
Smart devices such as smartphones and smartwatches today have many different sensors, often with high frequency and accuracy. One sensor that is becoming more common in smart devices is the barometer, this allows for air pressure to be measured, making altitude calculations possible. Since skydiving equipment is typically expensive, it would be useful for skydivers to enhance their skydiving experience using the smart devices that they already own. This article will discuss how an app might be created in order to give guidance to a skydiver to improve their canopy piloting.
\end{abstract}

\section{Introduction}\label{sec:introduction} % "Lays out the whole paper"

% What is the problem?

% Why is the problem important?

% What has been done so far on the problem?

% What is the contribution of this paper to the problem?

% Is the contribution original? Explain why

% Is the contribution non-trivial? Explain why

% Short summary of the rest of the paper
The rest of the paper is structured as follows: In Section~\ref{sec:related-work} we\dots


\section{Related Work}\label{sec:related-work}
\subsection{Apps}\label{sec:apps} % Existing apps, what they accomplish, how they do it, what they do well, what they don't do well (solved by me!)

%% L/D Vario %%
% Built for gliding airsports.
% Offers details on glide ratio, lift-to-drag ration, altitude, horizontal and vertical speed.
% Advanced version of the smart altimeter app.
% Uses a device's barometer to make calculations relative to the air.
% "Advanced multiple-layer algorithms are used to calculate smooth descent rate and vertical acceleration from noisy altitude data."
% The device needs to be mounted away from the body in undisturbed air stream.
% Highly recommends the use of devices with barometric sensors of 60Hz or faster
The~\citetitle{pfm_technologies_llc_l/d_2015} app by~\textcite{pfm_technologies_llc_l/d_2015} is an app that is built for ``gliding airsports''. The app allows for users to view their glide and lift-to-drag ratios, altitude, horizontal and vertical speeds among other things. The app seems to be built upon the much simpler~\citetitle{pfm_technologies_llc_smart_2016} app from the same author that only offers altitude and descent speed readings.
The~\citetitle{pfm_technologies_llc_l/d_2015} app uses the handheld or wearable smart device's barometer to make calculations relative to the air around it that result in the device's current altitude from either sea level or a user-defined reference point. Data important to wingsuiters is calculated with the help of the device's accelerometer. It is recommended that devices with barometric sensors of 60Hz or higher are used to reduce latency in altitude calculations. The authors also recommend that the device be mounted away from the body, at least 2--3 feet from a wingsuiter's wings, this is presumably to reduce the risk of the wings or body affecting the pressure readings in the device.

%% BlueFlyVario %%
% App designed to read data from the BlueFlyVario bluetooth pressure sensor from https://www.blueflyvario.com/
% Designed for for paragliding and hang gliding.
% Shows data such as altitude, wind speed, heading, flight time.
~\citetitle{dickie_blueflyvario_2016} is another app designed for gliding airsports, by~\textcite{dickie_blueflyvario_2016}. The app is designed to read data from the~\citetitle{noauthor_blueflyvario_nodate}~\cite{noauthor_blueflyvario_nodate} pressure sensor, providing data such as altitude wind speed, heading and flight time.
The app's use of an external barometric sensor is useful since it ensures that results across all devices running the app will be consistent and that the required hardware specifications of the barometer are met.
While the app does offer plenty of useful information, the user interface does seem cluttered and not user friendly. Since we do not have the~\citetitle{noauthor_blueflyvario_nodate} pressure sensor, we could not test any features of the app or sensor, however, with the website claiming ``altitude resolution of 10cm''~\cite{noauthor_blueflyvario_nodate} it can be seen why it would be desirable.

%% SpotAssist %%
% A tool for skydivers.
% Shows wind conditions at drop zones, safe plane exit area and recommended landing pattern based on changeable parameters (wind direction/speed, landing direction)
\citetitle{inc_spot_2017}~\cite{inc_spot_2017} is a different kind of app, focussing on pre and post-skydive assistance. Features of the app include wind condition forcasting, safe plane exit area calculating, cutaway finder and a landing pattern simulator. The wind condition forecasting shows information for different altitudes, information that is not typically available on standard forecast websites. The cutaway finder feature allows users to say where they cut away their main canopy so that the app can make an estimate of where the canopy may have landed based on wind speed and direction. The two likely most useful features for learner skydivers or skydivers at a new drop zone are the safe exit and landing pattern recommendations. The app is able to show a satellite image of drop zones and overlay recommendations on where to turn to land or where to exit the plane in order to be able to land in the target area. The app does not offer any features to feed exit or landing pattern data to skydivers during a jump and many features are restricted by a one time purchase or subscription.

\subsection{Barometers}\label{sec:barometers} % Explain findings from the barometer papers

% (Barometer Specs) Pressure/temp ranges, sampling rate, accuracy
Manufacturers of smart devices do not often advertise the make and model of sensors used in their devices, and rarely the specifications of them. \citeauthor{bosch_bmp280:_2016} manufacture the \textit{BMP280}, an absolute barometric pressure sensor chip that is advertised for being appropriate for use in smartphones. Absolute barometric pressure sensors measure the air pressure relative to a perfect vacuum. The data sheet for the chip~\cite{bosch_bmp280:_2016} states that the sensor has a pressure range of \SIrange{300}{1100}{\hPa}, equivalent to \SIrange{+9000}{-500}{\metre} above/below sea level. Since skydiving often takes place well below this height, the chip should be able to output altitude data during a typical skydive. The accuracy of the barometer readings within pressure ranges of \SIrange{950}{1050}{\hPa} at a temperature of \SI{25}{\degreeCelsius} according to the data sheet are \SI{\pm0.12}{\hPa}, which is equivalent to \SI{\pm1}{\metre}. One problem arises when we look at the ``absolute accuracy'' which is listed for the same pressure ranges but from \SIrange{0}{40}{\degreeCelsius}, an accuracy of \SI{\pm1}{\hPa} (\SI{\approx\pm8}{metre}). The absolute accuracy could pose to be an issue while skydiving since the value is for \SIrange{0}{40}{\degreeCelsius} and temperatures at skydiving altitudes can easily drop below zero, possibly resulting in a worse accuracy. \citeauthor{bosch_bmp280:_2016} lists the \textit{BMP280}'s typical sampling rate as \SI{182}{\Hz}, with an average terminal velocity while skydiving belly-to-earth of \SI{54}{\metre\per\second}, this sampling rate is enough to cover the descent rate.

% (Bluetooth transceivers)
While many smartphones and smartwatches do now come with barometric sensors built in, a large portion of manufacturers do not see them as being worth putting in their devices. Solutions do still exist for accurately getting barometric pressure readings with such devices however, as shown by~\textcite{he_atmospheric_2012}. The paper from \citeyear{he_atmospheric_2012} was made at a time when few smartphones featured barometric pressure sensors, the paper instead used a bluetooth connected circuit board with pressure sensing capabilities. While the barometric pressure sensor used by~\textcite{he_atmospheric_2012} ran at a very low sampling rate (one that would not be practical for skydiving applications), more advanced boards do exist as shown in Section~\ref{sec:apps} with the~\citetitle{noauthor_blueflyvario_nodate}.
The~\citetitle{noauthor_blueflyvario_nodate} is built for gliding airsports such as paragliding and boasts a sampling frequency of \SI{50}{\Hz} with ``a resolution that enables the measurement of altitude differences as small as \SI{10}{\cm}.''

% (Altitude calculation) Calculation for altitude from barometric pressure readings and temperature
    % Two similar calculations for altitude in the papers. Paper A gives ... Paper B gives...
    % One calculation a base HEIGHT and calculates current altitude on top of that, other calculates height directly from pressure readings.
    % I will be calculating all heights relative to the air pressure on the ground and so will not be making calculations relative to known object heights. Algorithm * will be used.
Since there is a relation between air pressure and altitude, altitude can be calculated from air pressure measurements. \textcite{liu_beyond_2014} defines an equation for converting air pressure to altitude. The equation requires a reference pressure value and the pressure at which elevation is to be calculated as well as some constants such as gravitational acceleration. The paper specifies that the reference pressure would be the standard atomospheric pressure which reflects air pressure at sea level. We can use this equation for calculating altitude while skydiving however for accuracy and ease of calculation purposes, it would be a good idea to calibrate the reference pressure to the pressure reading at ground level just before a skydive, since this will be the most recent and localised reading. Using the local ground pressure for reference in the equation would also remove the need to calculate the height of that position from sea level and subtracting it from all future calculations.

% (Temperature) The effects of temperature on barometer readings
    % Calibration of smartphones paper mentions integrated temperature sensor, not accessible via Android APIs. Inbuilt calibrations based on temp.
    % Temperature causes change in pressure readings. Can see in Fig 9 that as temperature rises on Galaxy Nexus, so too does pressure reading.
    % One might pressume that a decrease in temperature may result in a decrease in pressure readings.
In a paper investigating the calibration of smartphone sensors for indoor positioning~\cite{keller_calibration_2012}, tests were conducted to investigate the effects of temperature change on measurements from a smartphone's barometer. The \textit{BMP280} barometer has a temperature sensor that is used for applying inbuilt calibrations based on the measured temperature. While the barometer does have its own temperature dependent calibration, we can see from~\citeauthor{keller_calibration_2012}'s results that temperature changes do still have a noticeable effect on the device's pressure readings. Results showed a correlation whereby as temperature increased, so too did pressure, an unexpected result when we consider that air becomes less dense at higher temperatures. Inconsistencies in pressure readings between temperatures likely vary depending on the smart device. The paper only used one device, it would be interesting to investigate how much different barometers are affected by temperature to understand how much they can be relied on in quickly changing temperatures such as while skydiving.


% (Accuracy testing) Tested accuracy of barometer using building floors


% (Inaccuracies) Inaccurate when ascending or descending at rapid rates

% (Data filtering) Filtering burst and random errors using kalman filter

\subsection{GPS}\label{sec:gps} % Explain findings from barometer papers

% Horizontal accuracy

% Vertical accuracy about 2x worse than horizontal

\subsection{Bluetooth}\label{sec:bluetooth} % Explain findings from bluetooth papers

% Classes of bluetooth and their range/speed

\subsection{Wi-Fi Direct}\label{sec:wifi-direct} % explain findings from wi-fi direct papers

% Range and speeds of Wi-Fi direct

\section{Theory}\label{sec:theory} % Describes underlying theory behind the system (What, relation, definition, example)

% Altitude calculation

% Kalman filter (why use this)

% Canopy flight characteristics

\section{Specification}\label{sec:specification} % Work proposed

% Techniques that underlie the implementation

    % Learner skydiving safety requirements

    % Landing pattern point calculation method

% Requirements of the implementation

    % Able to show a recommended landing pattern based on wind speed and direction

    % Able to give instruction for landing pattern turn points based on real-time progress in order to land in the landing area (via bluetooth earpiece)

    % Able to communicate with other users of the app to warn of nearby canopies

\section{Results and Discussion}\label{sec:results-discussion} % Tests done and planned

% Discussion of results and evaluation

% Discussion of justification for work

% Discussion of experiments planned with justification

% Discussion of expected evaluation approach (final experiments)

\section{Conclusion} % Summarises research, discusses its significance

% Briefly restate project description

% Original motivation repeated

% State of the field in light of this paper reassessed

\printbibliography{}
\end{document}
