% !TEX root = ../main.tex
\section{Introduction}\label{sec:introduction}
% What is the problem?
Learning to skydive is a steep learning curve and every step of the way, a learner skydiver is responsible for their safety and, at least in part, for the safety of others. Each step of a skydive has its dangers which is why the whole of a skydive should be pre-planned. In particular, the journey under canopy (parachute) is arguably the most dangerous part of the skydive. A skydiver must ensure that they avoid all obstacles, such as other skydivers, guide themselves to a safe landing area and avoid making any dangerous low turns. Student skydivers often struggle to predict their trajectory while planning their landing pattern and this can result in a landing which is far from the intended landing area or taking dangerous actions in an attempt to meet their target area.

Skydiving equipment is costly; there are no tools beyond radio communication with an instructor on the ground to aid students with their canopy skills. However, this is usually not used after the first two jumps. An instructor watching a skydiver's landing from the ground may not be able to provide useful, detailed feedback on the skydiver's landing since distance and height are hard to gauge by eye from a distance. The problem that is faced by new skydivers is that there are not enough quality tools that are easily accessible to them to aid in their learning.

% Background knowledge to understand the problem
People skydive to get their thrill, and each person has a preference for their activity in the sky. Some people like to focus solely on their canopy flight, possibly doing formation drills with other canopy flyers, flying in very close proximity. Other skydivers enjoy the freefall of their skydive most, possibly alongside other skydivers. These different skydiving styles will have differences in the length of their freefall and canopy periods.
In any case, a skydive can be split into two sections:
\begin{enumerate}
  \item Freefall, immediately after exiting the plane.
  \item Canopy flight, once the skydiver successfully deploys their parachute.
\end{enumerate}

% Why is the problem important?
Improving learning tools available to skydivers is paramount because there are substantial safety concerns when skydiving students are flying their canopies, since they may act unpredictably or in a dangerous manner that can be hazardous to themselves or others. This danger can extend to experienced skydivers that may not be familiar with wind conditions or the landing pattern for an unfamiliar dropzone or climate.
For a skydiver to improve their skills, they must either remember what they did previously or have tools to document their skydive, such as having an instructor record the jump with a camera. Tools that aid in the revision of a jump are costly, sometimes hundreds of pounds.~\cite{dekunu_dekunu_????} The example of having an instructor record the jump would for example typically cost more than the price of another skydive.
Having tools readily available to skydivers at no extra cost could prevent higher numbers of accidents by preparing and improving skydiving knowledge.

% What has been done so far on the problem?
There has been very little done to address the problem of helping student skydivers to improve their landing patterns and be safer in the sky; all of the support takes place on the ground with rough, estimated feedback. Any data tracked accurately during a skydive is typically from an expensive altimeter that will only display basic statistics from a jump (skydive).

% Aim of our work
The work conducted for this paper aimed to create a proof of concept app called ``AccuDrop'' to aid skydivers before, during and after a skydive, focussing primarily on features to improve landing patterns and overall safety. We investigated the appropriateness of today's smart devices such as smartphones and smartwatches for use in fast-paced skydiving applications.

% Is our contribution original? Explain why
Through our research, we have found that this problem does not appear to have been tackled before. A few apps exist already for gliding air sports such as paragliding, but these only provide simple statistics such as current altitude and speed. Nothing could be found to provide live feedback or safety-enhancing features likely to be of much assistance to skydivers.

% Is our project non-trivial? Explain why
Android devices are not typically made to accommodate extreme sports such as skydiving. One of the challenges of the project was working to make the app's features as robust and reliable as possible. Creating a solution to calculating a safe and correct landing pattern also posed complications, along with Android restrictions and inadequacies. We had to create the whole project in a way that could be useful to a user but also have the latency and accuracy required for accurate tracking of a skydive.
