% !TEX root = ../main.tex
\section{Implementation and Testing}\label{sec:implementation-and-testing}
In this section, implementation details of the app's features are stated, including implementation of algorithms, data structures and justification for main decisions.

% Implementation of algorithms
% Data structures used
% Implementation decisions taken and justification

\subsection{Model View Processor}\label{subsec:mvp}
Activity and Fragment classes are what we would consider as the View in our MVP design pattern.
Android will recreate Activities and Fragments whenever the app rotates or switches the current screen being displayed. Due to this behaviour, more work is required to allow data to persist across these events so that views and running processes are not reset.

Android offers `\citetitle{_android_????}' libraries to help with managing the Android lifecycle.
We decided to use these libraries to form the MVP structure of the app's codebase, allowing for the `Model' elements containing data and data access methods to simply extend the `\texttt{ViewModel}' class and persist for the entire lifecycle of the view to which it is attached.

The architecture components library also adds the `\texttt{LiveData}' class which can encapsulate any other type. By returning data from our \texttt{ViewModel} classes in \texttt{LiveData} objects, we are given the ability to subscribe to these data objects so that a function we define is run when the value changes. We utilise this ability to subscribe to \texttt{LiveData} values in presenter classes to run back-end logic or to update our views based on the new values.

\subsection{Location Data}\label{subsec:location-data}
% GNSS
GNSS data is collected in our `\texttt{GnssListener}' class, this gets the system location service and registers itself as a location listener. The \texttt{GnssListener} is created by a \texttt{GnssViewModel} that passes itself to the listener as a constructor parameter, this allows for the listener to store its location data in the \texttt{GnssViewModel} that might then be accessed from other parts of the app.

Collecting an Android device's GNSS location requires the ``\texttt{ACCESS\_FINE\_LOCATION}'' permission. Once a \texttt{GnssListener} is instantiated, it checks whether the app has been granted the appropriate permission if it has not then a prompt is raised for the user to grant or deny the permission for the app. If the location permission is denied, then the user will be prompted again, this time with an explanation of why the app requires the permission, at this point the user can select not to be asked again, and then the feature requiring location permissions will be disabled.

Initially positioning data collection was implemented by registering a `\texttt{FusedLocationProvider\-Client}', Google recommends extending this class for fast and accurate location data. However, it was realised that this data might not be recent or accurate since it is a stored location from Google that may use cell tower or Wi-Fi/Bluetooth devices to triangulate a device's position. Location data has been known to be inaccurate with cell tower, Wi-Fi and Bluetooth location data due to the long range of cell tower signals or movement of Wi-Fi and Bluetooth signals. Due to the requirement of high precision location data, the decision was made to move to GNSS location data only.

% Altitude
Altitude data is collected in a similar way to GNSS data. A `\texttt{PressureViewModel}' is created to store and provide access to pressure/altitude data, and this creates a `\texttt{PressureListener}' that registers itself with the device as a sensor listener (specifically with the pressure sensor).

As with the \texttt{GnssListener}, the \texttt{PressureListener} sets its data in the \texttt{PressureViewModel} so that other parts of the app can listen/act on the data. There is a small amount of logic in the \texttt{PressureViewModel} class that calculates and stores the altitude that can be calculated from the pressure reading it is given.

  While designing the app, it was decided that we would use the international barometric formula shown in Equation~\ref{eq:barometric-fomula} in Section~\ref{subsec:altitude-calc-design} to calculate altitude. We found that Android already has an implementation of this equation in their `\texttt{SensorManager}' class, shown in Listing~\vref{lst:getAltitude}, thus we can reduce the code that we require.

\begin{listing}
  \centering
  \begin{minted}{java}
public static float getAltitude(float p0, float p) {
    final float coef = 1.0f / 5.255f;
    return 44330.0f * (1.0f - (float)Math.pow(p/p0, coef));
}
  \end{minted}
  \caption{Code from Android's SensorManager class to calculate altitude from two pressure values.~\cite{_android_????-1}}\label{lst:getAltitude}
\end{listing}

\subsection{Background Threads}
To avoid hanging the UI thread, making the app unresponsive due to long operations, we start long operations on new threads that run in the background. Long tasks are placed in classes that extend the Android `\texttt{AsyncTask}' class. AsyncTasks go through four stages, these are:
\begin{itemize}
  \item \texttt{onPreExecute} --- Run on the UI thread for task setup code for the task.
  \item \texttt{doInBackground} --- Run on the background thread for running code that can take a long time to execute.
  \item \texttt{onProgressUpdate} --- Run at any time \texttt{publishProgress()} and used to update other parts of the app of task progress such as by updating a progress bar.
  \item \texttt{onPostExecute} --- Run on the UI thread once the background thread code has completed.
\end{itemize}

The default Android behaviour for the `\texttt{execute()}' method run on an \texttt{AsyncTask} adds the task to be run on a thread pool containing only one thread; thus tasks are run synchronously. This behaviour did initially cause a lot of major performance issues in the app since the default behaviour was previously to run many threads in parallel and so this is what we were expecting. AccuDrop has been programmed in a thread-safe way; we were able to increase performance by instead using the `\texttt{executeOnExecutor()}' method with a larger thread pool to run tasks in parallel.

\subsection{Landing Pattern Planner}
% Overlay on a Google Map
Since Android already has built-in support for Google Maps, it was decided that the landing pattern planner screen would be filled by a Google Map on which we could overlay a landing pattern. An Android Google Map will also move the overlay with the map and allow for us to programmatically move the camera position.
When creating the Google Map, we must pass a callback method that will set up the map once it is ready for modification from the system. Overlays elements can be added using methods such as `\texttt{addMarker()}' for labelled markers and `\texttt{addPolyLine()}' for lines.

% Calculate a landing pattern for current location if that data is available
When the user opens the landing pattern planner screen, if location data is already available for the user, a landing pattern will automatically be generated with the target being the current location. When a pattern is created, we move the camera using the \texttt{CameraPosition} class to centre on the pattern with an appropriate level of zoom to fit the entirety of it.

% User can tap and hold to select a target landing area
If a user wishes to set their target landing area, they can do this by tapping and holding on a location on the map. This feature is possible since we have registered a long click listener for the map using `\texttt{setOnMapLongClickListener()}' that can initiate the logic required to calculate another route.

% How we calculate the landing pattern
\subsubsection{Generating a New Landing Pattern}
To calculate a landing pattern, we first set the target landing area coordinates stored in a `\texttt{LatLng}' object (consisting of latitude and longitude), in our \texttt{RouteViewModel} object. This object will contain all data relating to a landing pattern generated for a user.

Next, we instantiate a new `\texttt{WindTask}'. A \texttt{WindTask} is an \texttt{AsyncTask} that takes a callback method, a landing pattern presenter object and an OpenWeatherMaps API key. In the \texttt{onPreExecute} method of the \texttt{WindTask}, the presenter is informed that the task has begun where the presenter then starts a scrolling progress bar. In the \texttt{doInBackground} method of the \texttt{WindTask}, wind data is fetched from the OpenWeatherMaps API; this is further explained in Section~\vref{subsubsec:fetching-wind-data}. Finally, in \texttt{onPostExecute}, the wind speed and direction are extracted from the JSON response from \texttt{doInBackground}, this is then passed as a \mintinline{java}{Pair<Double, Double>} (a tuple of doubles) to the task's callback method. The presenter is then told that the task has finished so that it can remove the progress bar.

The \texttt{WindTask} callback method takes a pair of doubles that are the wind speed and direction and sets these in the \texttt{WindViewModel}, used to store wind data. Next, a new \texttt{RouteTask} is created and executed that takes as arguments another callback method and the wind data that was fetched.

The \texttt{RouteTask} will simply create a new \texttt{RouteCalculator} object that will calculate a landing pattern given wind data and a target as explained in Section~\vref{subsubsec:pattern-calculation}. Once a landing pattern has been calculated, this will be passed to the callback method which then sets the route in the \texttt{RouteViewModel}.

\subsubsection{Fetching Wind Data}\label{subsubsec:fetching-wind-data}
% Explain doInBackground of WindTask
To fetch weather data from OpenWeatherMaps, we craft a request URL specifying the target latitude, longitude, unit of measurement and API key. Listing~\vref{lst:api-request} shows how this URL is created in code.

\begin{listing}
  \centering
  \begin{minted}{java}
String api = "https://api.openweathermap.org/data/2.5/";
String url = api +
      "weather?" +
      "lat=" + latLng.latitude +
      "&lon=" + latLng.longitude +
      "&units=metric" +
      "&appid=" + apiKey;
  \end{minted}
  \caption{Crafting an OpenWeatherMaps API request url}\label{lst:api-request}
\end{listing}

Once a connection to the URL has been established, the response from the API is read from the InputStream from the connection, this response is JSON data containing weather information for the requested location.

An example JSON response is shown in Listing~\ref{lst:json-response}. Once we have read in all of the response, we then pass this to the `\texttt{JSONObject}' class which parses the JSON and returns a \texttt{JSONObject} object that we can make JSON queries to. The whole of the JSON response is one \texttt{JSONObject}, within this we have more \texttt{JSONObjects} such as ``\texttt{coord}'' and ``\texttt{weather}'' shown at the top of Listing~\ref{lst:json-response}, the object that we want is fetched by calling \mintinline{java}{responseObject.getJSONObject("wind")}. To fetch the values from the wind object, we call \mintinline{java}{windObject.getDouble("speed")} and \mintinline{java}{windObject.getDouble("deg")} to retrieve the wind speed and direction.

\begin{wrapfigure}{R}{0.45\textwidth}
  \begin{minted}{json}
{
  "coord": {
    "lon": -1.93,
    "lat": 52.45
  },
  "weather": [
    {
      "id": 521,
      "main": "Rain",
      "description": "shower rain",
      "icon": "09d"
    }
  ],
  "base": "stations",
  "main": {
    "temp": 10.24,
    "pressure": 989,
    "humidity": 100,
    "temp_min": 9,
    "temp_max": 11
  },
  "visibility": 10000,
  "wind": {
    "speed": 5.7,
    "deg": 150
  },
  "clouds": {
    "all": 75
  },
  "dt": 1522848000,
  "sys": {
    "type": 1,
    "id": 5055,
    "message": 0.0031,
    "country": "GB",
    "sunrise": 1522820049,
    "sunset": 1522867681
  },
  "id": 2650236,
  "name": "Edgbaston",
  "cod": 200
}
  \end{minted}
  \captionof{listing}{An example JSON weather response from the OpenWeatherMaps API}\label{lst:json-response}
\end{wrapfigure}

\subsubsection{Pattern Calculation}\label{subsubsec:pattern-calculation}
% Explain RouteCalculator.calcRoute() onwards
The \texttt{RouteCalculator} class contains a \texttt{calcRoute()} and several helper methods to calculate a landing pattern. The \texttt{calcRoute()} method returns a list of `\texttt{Location}' objects; these objects have several variables that can be set such as altitude, latitude, longitude and speed. Each location contains latitude, longitude and altitude to represent a location in the calculated landing pattern.

The \texttt{calcRoute()} method calls helper methods \texttt{calcP3()}, \texttt{calcP2()} and \texttt{calcP1()} to calculate each point in the landing pattern in reverse, as designed in Section~\ref{subsec:landing-pattern}. Helper methods to calculate the time to travel between altitudes and coordinates after a move of a distance in a bearing are also implemented as designed in their respective sections within Section~\ref{sec:design}. The only difference is that in the method to calculate a move of a distance in a bearing, the distance is input as metres and then converted to kilometres.

\subsubsection{Updating the Map}
% Subscribed to route in RouteViewModel
When a new route has been set in the \texttt{RouteViewModel}, the landing pattern presenter is given the new route since it is a subscriber to the data. The presenter then sends the new landing pattern to the view to update the displayed map overlay.

The map overlay is cleared, and a line is then created for a path between each point in the landing pattern using ``new \texttt{PolylineOptions}'', and ``\verb|add({coordinateA}, {coordinateB})|'', these lines are then added to the Google Map using the \texttt{addPolyLine()} method.

\subsection{Jump Logging}
When logging jump data, we first add a record for a new jump in the `Jump' table and then add position entries when a jump is in progress to keep positional data for review later.

In the design, we said that logging could begin if the user is above a safe skydiving altitude and if they reach a specific fall rate. We can check the user's altitude using the methods stated in Section~\ref{subsec:location-data}. To test the fall rate, a simple method to take two pressure measurements and two timestamps was created to calculate the fall rate of the user.

% Database
The Android Architecture Components libraries previously mentioned in Section~\ref{subsec:mvp} also offers the `Room Persistence Library', this allows for SQLite database features and offers compile time error checking in SQL queries.

The Room library was chosen for the implementation of the database since it abstracts away from the actual database technology and provides safe, checked access to a database from the app as well as ensuring that compatibility has been added by the programmer between different database schemas.

% Can return LiveData
Another benefit of using the Room library is that we specify that queries will return \texttt{LiveData} objects, allowing for parts of the app to subscribe to updates to database query results.

Each table in the database has a `DAO' (Data Access Object) that specifies the methods that are generated to give access to the app to make SQL queries to the database. For example, there are over twenty methods to represent various queries for the position table, from getting the latest jump number to retrieving every position for a jump.

DAO methods are defined with an \texttt{@Annotation} that will be processed at compile-time. There are several annotations available, but the ones that we used were mainly \texttt{@Insert} for inserting an entry and \texttt{@Query} for making a query to the database. Since there is the possibility that a database query could take a long time, a rule is enforced that restricts us from making queries on the UI thread that could make the UI unresponsive. All interactions with the database must be made on a different thread, such as in an \texttt{AsyncTask}.

Testing the logging of a jump was challenging to do since we did not have anyone doing any skydives with a compatible device during the project until during the writing of this report. Instead, pressure and GNSS data were logged while on the ground, this revealed that logging did indeed work. One problem that was found was that initially, the database entries required every piece of data which meant that if there no satellite lock, for example, data would not be logged. This issue was solved by changing position entries to not require all data.

\subsubsection{Ensuring Data is Logged}
Ideally, we would like to switch to a real-time process scheduler such as \texttt{SCHED\_FIFO} so that our process cannot be pre-empted and can perform its best. While this is possible, it would require root access to the phone, something that many users do not have.

It is possible to increase thread priority on Android which could help improve performance for our app. However, this has been known not to be honoured by many flavours of Android and so time was not put into making this change.

A `foreground service' was created to handle the logging of jump data. This service can run in the background, even if the device is locked because the service is treated as if it is a foreground app element, it is also not subject to being killed for memory and power saving reasons.

\subsection{Proximity Warnings}
% Wi-Fi Direct
As planned, communication for the proximity warnings was implemented using Wi-Fi Direct. Android allows for Wi-Fi direct functionality through its P2P (Peer-to-Peer) API~\cite{_creating_????}. However, the requirements to make and handle connections via the API are not trivial or streamlined. A `\texttt{BroadcastReceiver}' that we extend must be registered through the `\texttt{WifiP2pManager}' system service; this handles different changes in the states of the Wi-Fi connection interface. In particular, our registered \texttt{BroadcastReceiver} reacts to a change in the list of peers and informs our `\texttt{Peer2Peer}' object to attempt to connect to one of the peers.

What had not been realised until the Wi-Fi Direct functionality had been implemented is that any and all Wi-Fi Direct devices will show up in the list of peers, meaning that the app might attempt to communicate with for example a printer. We needed a way to identify signals that are from a device running the app. One consideration was that we could adjust change the device's Wi-Fi Direct name and look for this in the peer list, this may, however, upset users that have already set up their preferred device name.

The solution found for identifying Wi-Fi Direct signals running the AccuDrop app was to use DNS Service Discovery. P2P classes were altered to register a service with the name ``\texttt{accudropproximity}'' and a service type of ``\texttt{presence}'' which is defined for peer-to-peer messaging. We were not able to find a generic service type, and so the simple chat type was kept. A list of service types is available on the DNS-SD website~\cite{_dns_????}.

Once the app has registered its service, it then registers a listener for service discovery; if a service is found with the same service name, the app attempts to make a connection to it. The devices will decide between each other which device should be the ``owner'' and which should be the ``peer'', a process that is hidden from us in the operating system. Hopefully, a connection is successfully established and then each device either opens a server socket if they are the owner or connects to the other device's socket if they are the peer. Both devices can now send and receive data over this socket. Since the P2P API supports groups, more devices can connect to the group owner to send messages through the network.

If a device is currently logging canopy flight, the foreground service will attempt to make these connections and send its location. Upon receiving location data from another device, the app will place this data in a `\texttt{Location}' object and then use the \texttt{distanceTo()} method to get the distance from its own location. If the distance is below a configurable warning distance, text-to-speech is used to warn the user.

Text-to-speech is straightforward to generate; a new `\texttt{TextToSpeech}' object is created with a lambda function to set the sentence to say if text-to-speech was initialised correctly. The code for generating text-to-speech can be seen in Listing~\vref{lst:tts}, the code flushes the queue immediately so that the user gets their warning almost immediately.

\begin{listing}
  \centering
  \begin{minted}{java}
tts = new TextToSpeech(this, status -> {
    if (status == TextToSpeech.SUCCESS) {
        tts.setLanguage(Locale.UK);
        tts.speak("Warning, user " + (int) distance + " " + unitString + " away.",
                TextToSpeech.QUEUE_FLUSH, null, null);
    }
});
  \end{minted}
  \caption{Code to start a text-to-speech proximity warning in the app}\label{lst:tts}
\end{listing}

Unfortunately, there do seem to be quite a few issues with the Android P2P system, these issues and frustrations have been reflected by many programmers in the various online support channels. Establishing a connection or finding other devices seems to be unreliable, we found that quite often, the system would fail in these respects.

Another issue that was found is that connections must be accepted on the peer device by the user. Ideally, we would like for the connections to accepted automatically in a way similar to paired Bluetooth devices so that if a connection is lost at any point from the initial connection on the ground, the connection could be re-established in any case.

The final problem that was found is with the flexibility of the service types; we could not find a service type that applied to our use-case of sending location information and so a simple messaging service was chosen. An attempt was made to send serializable location objects. However, the data never reached the other device once sent. We suspect that this may be a limitation of the service type and so location data is instead sent as a string and then parsed on the other end.

Initially, to test the communication between devices, the devices were kept next to each other and set up to send test messages to each other. Once the test messages were sending, the sending of GPS coordinates was tested. Messages sent between devices would be output into the logs so that we could see whether communication between the devices was working.

Final testing of the proximity warnings was done by having two people holding device each and ensuring that while connected, a proximity warning is created at roughly the correct distance. This test worked for a reasonably large warning distance however when the warning area was reduced; the system was less reliable due to inaccuracy in the GPS sometimes detecting distances as greater than they were.

\subsection{Jump Generation for Testing}
Unfortunately, due to the uncommon activity that the project focusses on, and the low prominence of smartphones with barometers, it was very difficult to get skydive data with the app. Until we had a logged skydive, we had to generate jump data instead and use this for testing the features of the app used for reviewing a skydive.

For jump generation, firstly, a new jump entry is added to the database. Then, a new landing pattern is calculated for a static coordinate with a random wind speed from \SIrange{0}{10}{\metre\per\second} and a random wind direction from \SIrange{0}{360}{\degree}.

Once a landing pattern has been calculated, we add intermediary positions along the pattern to simulate a higher number of measurements that we would expect when a landing pattern is being recorded.

\begin{listing}
  \centering
  \begin{lstlisting}[style=code]
for (i = 0; i < pattern.size() - 1; i++) {
    set loc1 to location i in pattern
    set loc2 to location i + 1 in pattern
    set totalDistance to loc1.getDistanceTo(loc2)
    set altitude to loc1.getAltitude()
    set split to totalDistance / 5 metres
    set altitudeDecrement to (altitude - loc2.getAltitude()) / split

    append loc1 to result
    set prevLoc to loc1
    for (j = 0; j < split - 1; j++) {
        set bearing to prevLoc.bearingTo(loc2)
        set randBear to randIntBetween(-15,15)
        set prevLoc to prevLoc.positionAfterMove(5 metres,
          bearing + randBear)
        set altitude to altitude - altitudeDecrement
        set prevLoc.altitude to altitude

        append prevLoc to result
    }
}
append pattern.getLastLocation() to result
  \end{lstlisting}
  \caption{Pseudocode for adding intermediary points to a list of locations in a pattern}\label{lst:intermediary-points}
\end{listing}

The distance between two points in the landing pattern is calculated and split into some 5-metre gaps. For each 5 metre gap, a new location is calculated that descends an even amount for the split count and that moves 5 metres in the direction of the next calculated landing pattern point plus or minus a random \SI{15}{\degree}. Each calculated landing pattern position and newly calculated position returned as the new jump data. This process can be seen in full in the pseudocode in Listing~\vref{lst:intermediary-points}.

Next, the data is mirrored on top of itself by reversing the list and incrementing the altitudes by the previously highest altitude plus \SI{30}{\metre}; this is what we treat as our freefall data. The fact that the large horizontal movements are not realistic for freefall does not matter since we are only testing the ability to replay the data reliably primarily at this point.

The final step is to add the generated jump positions to the database, or if there are guest entries to be added (purpose explained in Section~\vref{subsec:formation-review}, the process begins again for the guest count, randomising the data each time.

This generated data can be used to test landing pattern review features in a realistic way since lots of jump positions are generated, and they are generated with noise on the route of an expected level.

\subsection{Landing Pattern Review}
For the landing pattern review feature of the app, we decided to once again place an overlay on a Google Map, only this time with jump data fetched from the database. Unfortunately, Google maps do not offer the ability to set overlay positions in 3D space, despite being able to adjust the map's camera height.

Due to this unforeseen restriction, a new view was decided to be implemented on the same screen view that can draw the perpendicular projection of a landing pattern, updated to match the orientation of the Google Map.

The app will rotate 3D coordinates consisting of the user's logged latitude, longitude and altitude.
The altitude will always be scaled across the Y coordinates of the screen, and so we do not need to modify this data. Initially, orthographic projection algorithms were considered for moving the user's viewpoint around the landing pattern. However, these would require far more calculation than is required for the feature. Instead, we rotate the landing pattern coordinates using a rotation matrix and keeping the user's view position constant.

Passing the bearing from the Google Map and each landing pattern coordinate into Equation~\vref{eq:matrix-mult}~\cite{lounesto_clifford_2001} gives us the new rotated coordinates that we use for the landing pattern perpendicular projection.

\begin{equation}\label{eq:matrix-mult}
  \begin{bmatrix}
    \mathrm{newX} \\[0.3em]
    \mathrm{newY}
  \end{bmatrix}
  =
  \begin{bmatrix}
    \cos(\theta)  & -\sin(\theta) \\[0.3em]
    \sin(\theta) & \cos(\theta)
  \end{bmatrix}\cdot
  \begin{bmatrix}
    x \\[0.3em]
    y
  \end{bmatrix}
\end{equation}
where
\begin{description}
  \item $\theta$: the map rotation in radians
  \item $x$: the longitude
  \item $y$: the latitude
  \item $\mathrm{newX}$: the resulting longitude
  \item $\mathrm{newY}$: the resulting latitude
\end{description}

To get the correct orientation of the newly rotated landing pattern, we now take the longitude as the screen X coordinate and the altitude as the screen Y coordinate.

Due to the differing sizes of device screens, we cannot scale coordinates by a static amount. Android will only give the dimensions of the area that we want to draw in at the time that the area should be should be drawn; this means that before we loop through screen coordinates, drawing lines between them, we must generate those coordinates from the landing pattern, scaled to the display size.

A helper function was created that called ``getScaledValue'' that takes as arguments, a value to scale, the minimum value the input could be, the maximum value the input could be, the minimum output value to scale to, and the maximum output value to scale to. This function allows us to scale an input value from one range to another, in this case from altitude to screen height for example. The equation is as follows:
\begin{equation}\label{eq:scale-values}
  \mathrm{R} = \left((O_{\mathrm{max}} - O_{\mathrm{min}}) \times \frac{V - I_{\mathrm{min}}}{I_{\mathrm{max}} - I_{\mathrm{min}}}\right) + O_{\mathrm{min}}
\end{equation}
where
\begin{description}
  \item $V$: the value to scale
  \item $I_{\mathrm{min}}$: the minimum input range value
  \item $I_{\mathrm{max}}$: the maximum input range value
  \item $O_{\mathrm{min}}$: the minimum output range value
  \item $O_{\mathrm{max}}$: the maximum output range value
  \item $R$: the result
\end{description}

Generated jump data was used to test the feature. Since we know what landing patterns should look like for the landing pattern planner, when the feature is working we can see that the same patterns are shown as being logged but with simulated GNSS noise.

To test whether the perpendicular view is working, we can easily compare with the overlay on the Google Map. It was clear during testing when the coordinate systems for drawing on the screen were accidentally flipped since the patterns did not match between views.

\subsection{Formation Review}\label{subsec:formation-review}
While creating the app, it was decided that one useful feature that was missed would be the ability to show many user's jump data for a formation skydive to be able to review who may have been in the wrong formation position at a point in time.

User UUIDs were added to the app to give each instance its own unique identifying data; this means that different user's data can be queried from and stored in the same database table. Next, the jump generation code was modified to generate multiple users' jump data.

The formation review feature is designed to look like a radar screen, there is, however, one significant difference. The difference in altitude from the subject user is displayed by a line from their coordinates on the flat radar to another point above or below as a scaled difference in altitude.

The data to present on the radar view is gathered by querying the database for all users on the same jump; then for each user, the app uses a binary search to find the positional data closest to the time in a jump that was selected for the subject user. The subject time is set via movement of a seek bar representing time from the start of the jump to the end of the jump.

If the distance from the subject user's position to another user's position is over a maximum value (static in the code but could be made configurable), then the other user's position is ignored. All user's position data that is within range of the subject is stored in a list to be processed.

The list of positions to process is converted to a list of bearings, distances and altitude differences from the subject user. Now that the list of user positions has been converted to positions relative to the subject, they can be scaled for the screen relative to a central point representing the subject. Since the radar is an ellipse to make distance more apparent, the 3D coordinates' Z values are scaled to the change in the height of the ellipse from a circle.

All points placed on the screen are stored in a list. A listener has been registered so that if the user touches the screen, the list of points is searched to see if the user touched within an area of one of the points drawn on the screen. If the user touched within a defined region of a drawn point, the view is cleared and recalculated with the user that the point represents as the new subject user.

To test the radar formation review screen initially, jump data for three users was manually created with known positions; this allowed for us to be able to see whether users were being placed on the screen correctly relative to each other.

\subsection{Settings}
A settings menu was created to allow users to be able to adjust certain values within the app such as canopy airspeed and glide ratio, and landing pattern turn heights.

Modifying a setting can be presented to a user is several different forms, for example, a dropdown selection menu, number entry or a toggle. With the first version of our implementation, many of the settings were number entry only. We assumed that with number entry input, values would be automatically stored as integers or floats, this was however not the case. The standard way of storing preferences in Android seems to be in string form only.

Since we had preferences stored as numbers in string form and default preference values stored as integers, code required to fetch numerical values from preferences became more complicated.
Fetching a preference value requires as input the ``key'' that it was stored under and a default value. Since we wanted an integer output a lot of the time, this meant that our code had to convert between strings and integers more than we would like, as shown in Listing~\ref{lst:fetch-pref}. A lot more error checking was required because of these conversions.

\begin{listing}
  \centering
  \begin{minted}{java}
p1Altitude = Integer.valueOf(sharedPreferences
    .getString("landing_pattern_downwind_altitude",
      String.valueOf(
        resources.getInteger(R.integer.pref_default_downwind_altitude)
      )
    )
  );
  \end{minted}
  \caption{Code to fetch the value for an integer preference}\label{lst:fetch-pref}
\end{listing}

The settings screen uses Android \texttt{@TargetApi} annotations and support libraries to provide different functionality implementations to devices running different versions of Android.

Everything that accesses the app's settings/preference values registers an ``\texttt{OnSharedPreference\-ChangeListener}'' that handles any changes to the preference values to recalculate or change the display of one of the app's features. One example of where this feature can be seen is in the landing pattern planner, where each point in a landing pattern can be clicked on to reveal its altitude. If the unit of measurement preference is changed, the altitude will be recalculated into the new unit of measurement.

% TODO
% Testing strategy
% Examples of testing
% Relate tests to software requirements specification
