% !TEX root = ../main.tex
\subsection{Related Work}\label{subsec:related-work}

\subsubsection{Current Apps}\label{subsubsec:apps} % Existing apps, what they accomplish, how they do it, what they do well, what they don't do well (solved by me!)

%% L/D Vario %%
% Built for gliding airsports.
% Offers details on glide ratio, lift-to-drag ration, altitude, horizontal and vertical speed.
% Advanced version of the smart altimeter app.
% Uses a device's barometer to make calculations relative to the air.
% "Advanced multiple-layer algorithms are used to calculate smooth descent rate and vertical acceleration from noisy altitude data."
% The device needs to be mounted away from the body in undisturbed air stream.
% Highly recommends the use of devices with barometric sensors of 60Hz or faster
\paragraph{L/D Vario}\label{ld-vario}
The \citetitle{pfm_technologies_llc_l/d_2015} Android app by \textcite{pfm_technologies_llc_l/d_2015} is an app that is built for ``gliding airsports''. The app allows for users to view their glide and lift-to-drag ratios, altitude, horizontal and vertical speeds among other things. The app seems to be built upon the much simpler \citetitle{pfm_technologies_llc_smart_2016} app from the same author that only offers altitude and descent speed readings.
The \citetitle{pfm_technologies_llc_l/d_2015} app uses the handheld or wearable smart device's barometer to make calculations relative to the air around it that result in the device's current altitude from either sea level or a user-defined reference point. Data relevant to wingsuiters is calculated with the help of the device's accelerometer. The authors suggest that devices with barometric sensors of 60Hz or higher be used to reduce latency in altitude calculations. The authors also recommend that the device is mounted away from the body, at least 2--3 feet from a wingsuiter's wings (likely mounted from the chest), this is presumably to reduce the risk of the wings or body affecting the pressure readings in the device.

%% BlueFlyVario %%
% App designed to read data from the BlueFlyVario bluetooth pressure sensor from https://www.blueflyvario.com/
% Designed for for paragliding and hang gliding.
% Shows data such as altitude, wind speed, heading, flight time.
\paragraph{BlueFlyVario}\label{blueflyvario}
\citetitle{dickie_blueflyvario_2016} is another app designed for gliding air sports~\cite{dickie_blueflyvario_2016}. The app is designed to read data from a \citetitle{_blueflyvario_????}~\cite{_blueflyvario_????} pressure sensor, providing data such as altitude, wind speed, heading and flight time.
The app's use of an external barometric sensor is useful since it ensures that results across all devices running the app will be consistent and that they will meet the required hardware specifications of the barometer.
While the app does offer plenty of useful information, the user interface does seem cluttered and unintuitive. The app is also focussed toward gliding sports where horizontal movement is most important, as opposed to skydiving where significantly higher fall rates are reached. Since we do not have the \citetitle{_blueflyvario_????} pressure sensor, we could not test any features of the app or sensor, however, with the website claiming ``altitude resolution of \SI{10}{\centi\metre}''~\cite{_blueflyvario_????} it can be seen why it would be desirable.

%% SpotAssist %%
% A tool for skydivers.
% Shows wind conditions at drop zones, safe plane exit area and recommended landing pattern based on changeable parameters (wind direction/speed, landing direction)
\paragraph{SpotAssist}\label{spotassist}
\citetitle{inc_spot_2017}~\cite{inc_spot_2017} is a different kind of app, focussing on pre and post-skydive assistance. Features of the app include wind condition forecasting, safe plane exit area calculating, cutaway finder and a landing pattern simulator. The wind condition forecasting shows information for different altitudes which is not typically available on popular forecast websites. The cutaway finder feature allows users to say where they cut away their main canopy in an emergency so that the app can estimate where the canopy may have landed based on wind speed and direction. The two likely most useful features for learner skydivers or skydivers at a new drop zone are the safe exit and landing pattern recommendations. The app can show a satellite image of drop zones and overlay recommendations on where to turn to land or where to exit the plane to be able to land in the target area. The app does not offer any features to aid a skydiver during a jump, and many features are restricted, requiring a one-time purchase or subscription.

\subsubsection{Barometers}\label{subsubsec:barometers} % Explain findings from the barometer papers
\paragraph{Smartphones}\label{smartphones-barometers}
% (Barometer Specs) Pressure/temp ranges, sampling rate, accuracy
Manufacturers of smart devices do not often advertise the make and model of sensors used in their devices, and rarely the specifications of them. \citeauthor{bosch_bmp280:_2016} manufacture the \textit{BMP280}, an absolute barometric pressure sensor chip that is advertised as being appropriate for use in smartphones. Absolute barometric pressure sensors measure the air pressure relative to a perfect vacuum. The datasheet for the chip~\cite{bosch_bmp280:_2016} states that the sensor has a pressure range of \SIrange{300}{1100}{\hPa}, equivalent to \SI{+9000}{\metre} above to \SI{-500}{\metre} below sea level. Since skydiving often takes place well below this height, the chip should be able to output altitude data during a typical skydive. The accuracy of the barometer readings within pressure ranges of \SIrange{950}{1050}{\hPa} at a temperature of \SI{25}{\degreeCelsius} according to the datasheet are \SI{\pm0.12}{\hPa}, which is equivalent to \SI{\pm1}{\metre}. One problem arises when we look at the ``absolute accuracy'' which is listed for the same pressure ranges but from \SIrange{0}{40}{\degreeCelsius}, an accuracy of \SI{\pm1}{\hPa} (\SI{\sim\pm8}{\metre}). The absolute accuracy could pose to be an issue while skydiving since the value is for \SIrange{0}{40}{\degreeCelsius} and temperatures at skydiving altitudes can easily drop below zero, possibly resulting in a worse accuracy. \citeauthor{bosch_bmp280:_2016} lists the \textit{BMP280}'s typical sampling rate as \SI{182}{\Hz}, the average terminal velocity of a skydiver is \SI{54}{\metre\per\second}, for each sample the barometer takes, a skydiver would only move \SI{\sim0.3}{\metre}.

% (Bluetooth transceivers)
\paragraph{Bluetooth Transceivers}\label{bluetooth-transceivers}
While some smartphones and smartwatches do now come with barometric sensors built in, a significant portion of manufacturers still do not put them in their devices. Solutions do exist for accurately getting barometric pressure readings with such devices, however, as shown by~\textcite{he_atmospheric_2012}. The paper from \citeyear{he_atmospheric_2012} was made at a time when few smartphones featured barometric pressure sensors; the paper instead used a Bluetooth connected circuit board with pressure sensing capabilities. While the barometric pressure sensor used by~\citeauthor{he_atmospheric_2012} ran at a low sampling rate (one that would not be practical for skydiving applications), more advanced boards do exist as shown in Section~\ref{subsubsec:apps} with the \citetitle{_blueflyvario_????}.
The \citetitle{_blueflyvario_????} is built for gliding air sports such as paragliding and boasts a sampling frequency of \SI{50}{\Hz}.

% (Altitude calculation) Calculation for altitude from barometric pressure readings and temperature
% Two similar calculations for altitude in the papers. Paper A gives ... Paper B gives...
% One calculation a base HEIGHT and calculates current altitude on top of that, other calculates height directly from pressure readings.
% I will be calculating all heights relative to the air pressure on the ground and so will not be making calculations relative to known object heights. Algorithm * will be used.
\subsubsection{Altitude Calculation}\label{subsubsec:altitude-calculation}
Since there is a relation between air pressure and altitude, altitude can be calculated from air pressure measurements. \textcite{liu_beyond_2014} defines an equation for converting air pressure to altitude, stated in Section~\ref{subsec:altitude-calc-design}. The equation requires a reference pressure value and the pressure at which elevation is to be calculated as well as some constants such as gravitational acceleration. The paper specifies that the reference pressure would be the standard atmospheric pressure which reflects air pressure at sea level. We can use this equation for calculating altitude while skydiving however for accuracy and ease of calculation purposes; it would be a good idea to calibrate the reference pressure to the pressure reading at ground level just before a skydive since this will be the most recent and localised reading. Using the local ground pressure for reference in the equation would also remove the need to calculate the height of that position from sea level and subtracting it from all future calculations.

% (Temperature) The effects of temperature on barometer readings
% Calibration of smartphones paper mentions integrated temperature sensor, not accessible via Android APIs. Inbuilt calibrations based on temp.
% Temperature causes change in pressure readings. Can see in Fig 9 that as temperature rises on Galaxy Nexus, so too does pressure reading.
% One might presume that a decrease in temperature may result in a decrease in pressure readings.
\subsubsection{Temperature}\label{subsubsec:temperature}
In a paper investigating the calibration of smartphone sensors for indoor positioning~\cite{keller_calibration_2012}, tests were conducted to study the effects of temperature change on measurements from a smartphone's barometer. The \textit{BMP280} barometer has a temperature sensor that is used for applying inbuilt calibrations based on the measured temperature. While the barometer does have its temperature dependent calibration, we can see from \citeauthor{keller_calibration_2012}'s results that temperature changes do still have a noticeable effect on the device's pressure readings. Results showed a correlation whereby as temperature increased, so too did pressure, an unexpected result when we consider that air becomes less dense at higher temperatures. Inconsistencies in pressure readings between temperatures likely vary depending on the smart device. Their paper used only one device, investigating how different barometers' pressure readings are affected by temperature would be useful, especially for significant changes in temperature found in skydiving.

% (Accuracy testing) Tested accuracy of barometer using building floors
% Accuracy shown to be good in both Nengquiang and keller tests
% Accuracy not yet tested at high speeds and larger altitudes
\paragraph{Accuracy}\label{accuracy}
Smartphones have previously been tested for their barometric pressure sensing accuracy. In a paper by~\textcite{keller_calibration_2012}, researchers took a smart device to different floors of a building and recorded their results. The researchers found that the altitude data from the phones they used had an accuracy better than \SI{\pm1}{\metre}. This data is for small changes in altitude with slow rates of change; it is important that tests be carried out to discover the accuracy of smart device altitude measurements across large altitude ranges, temperatures and at faster speeds.

% (Inaccuracies) Inaccurate when ascending or descending at rapid rates
Inaccuracies can occur in smart device barometers. The temperature must be accounted for in calibration and positioning must be considered, as suggested by the \citetitle{pfm_technologies_llc_l/d_2015} app mentioned in Section~\ref{subsubsec:apps}. Characteristics of smart device barometers may not be suited to high ascent/descent rates. \textcite{gray_integrated_1995} claim that barometers are ``most inaccurate when ascending or descending at rapid rates (especially noted in fighter aircraft)''. A skydiver is not likely to be descending at similar rates of a fighter aircraft, but since smart devices are not built with extreme descent rates in mind, significant inaccuracies could be found that might not have been previously discovered through testing on floors of a building.

% (Data filtering) Filtering burst and random errors using Kalman filter
\paragraph{Data Filtering}\label{data-filtering}
Papers testing the use of barometers for calculating altitude have stated the existence of burst and random errors~\cite{gray_integrated_1995, liu_beyond_2014}. Extended Kalman filtering is used to successfully increase the accuracy of the pressure readings in~\textcite{liu_beyond_2014}, a paper investigating the use of smartphone barometers to improve the accuracy of elevation data. The Extended Kalman Filter linearises data according to an estimate based on the current mean and covariance of data that has already been seen~\cite{julier_unscented_2004}. The paper states that the elevation obtained by the ``Kalman filter is generally better than without'' the Kalman filter. While it may seem like a good idea to apply an Extended Kalman Filter to all smart device barometric data, we must be careful because our use of the data will be safety-critical. Extended Kalman filtering may not deal so well with rapid and fluctuating changes in air pressure that would come from skydiving. \textcite{huang_analysis_2008} state that poor initial estimates or incorrect modelling can lead to a divergence from expected results, this is not something that should be occurring during a skydive.

\subsubsection{GNSS}\label{subsubsec:gps} % Explain findings from barometer papers
% Horizontal accuracy (Understanding GPS: principles and applications)
% 4m horizontal accuracy best solution for getting longitude and latitude coordinates
Global Navigation Satellite System (GNSS) is currently the best solution for obtaining the longitude and latitude of a smart device outdoors. The horizontal accuracy for the new GALILEO GNSS system has a horizontal accuracy of just \SI{4}{\metre}~\cite{kaplan_understanding_2005} with a maximum public use precision of \SI{1}{\metre}. While the GALILEO system is reasonably accurate, we must remember that not all devices support the navigation service yet and thus may need to use other satellite services such as GPS that have worse accuracy.

% Vertical accuracy about 2x worse than horizontal
% 8m vertical accuracy not great, 2x worse than horizontal
% Not appropriate for usage in skydiving where quick/accurate altitude data required
\textcite{kaplan_understanding_2005} have shown the vertical accuracy of the most accurate GNSS to be approximately two times worse than horizontal, giving a vertical accuracy of \SI{8}{\metre}. Since the vertical accuracy of GNSS is lower than a smart device barometer and relies on a signal from satellites, it would not be wise to use it as a primary source for altitude data when barometric data is available.

\subsubsection{Bluetooth}\label{subsubsec:bluetooth} % Explain findings from Bluetooth papers

% Classes of Bluetooth and their range/speed
There are three classes of Bluetooth device, each with a different intended range. Class 1 Bluetooth devices have the farthest intended range of \SI{100}{\metre}; Class 2 devices can have range up to \SI{30}{\metre} or \SI{10}{\metre} depending on which source is read~\cite{_bluetooth_????, wright_dispelling_????} and Class 3 have the shortest range, less than \SI{10}{\metre}.
For skydiving applications, we would need device communication over the most extended range possible, unfortunately, smart devices typically only have Class 2 Bluetooth capabilities. While smart devices may not have long-range Bluetooth capabilities themselves, programmable Bluetooth beacons can be purchased that advertise ranges up to 400m~\cite{_coin_????}. Bluetooth 5.0 has promised to offer up to 4 times the range previously possible~\cite{bluetooth_sig_inc_rethinking_????}, and so it is likely that we will see more devices with long-range Bluetooth capabilities soon. Bluetooth beacons can be used to extend the Bluetooth capabilities of smart devices; however, this would be an additional purchase required for users which should be avoided if possible. With the addition of mesh network capabilities in Bluetooth 5.0~\cite{_mesh_????}, the effective range of Bluetooth could be increased through links between many devices.

\subsubsection{Wi-Fi Direct}\label{subsubsec:wifi-direct} % explain findings from Wi-Fi direct papers

% Range and speeds of Wi-Fi direct
Wi-Fi Direct can be used to achieve higher ranges and speeds than Bluetooth, the \citeauthor{wi-fi_alliance_wi-fi_????} state on their website that connections can exist in a range up to \SI{200}{\metre}~\cite{wi-fi_alliance_wi-fi_????}. The range of Wi-Fi Direct does depend on the devices being used, the same as would be expected for a standard Wi-Fi connection. Considering the specification of a maximum range that is double what is specified for Bluetooth, Wi-Fi Direct seems more appropriate for the communication between devices during skydiving activities.

% Number of devices
Unlike with Bluetooth 5.0, mesh networks are not available for Wi-Fi Direct devices; networks must be either one-to-one or one-to-many. For each skydiver's device to collect data from other nearby skydivers, it must either create a connection to each other device in turn or a single device should be nominated as the network host in a one-to-many relationship.
